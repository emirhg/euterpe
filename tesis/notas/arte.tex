\section {Arte}

\paragraph{
  "La obra es el origen del artista. Ninguno puede ser sin el otro. Pero ninguno de los dos soporta tampoco al otro por separado. El artista y la obra son en sí mismos y recíprocamente por medio de un tercero que viene a ser lo primero, aquello de donde el artista y la obra de arte reciben sus nombres: el arte" \cite{heidegger-1936}.
}

\paragraph{
  Para que algo sea considerado arte debe existir una apreciación subjectiva que le valorice cómo tal.
}

\paragraph{
  Al arte aunque se trate de una existencia efímera, existe en el mundo material y cómo tál está conformado por una configuración particular de elementos únicos y finitos.
}

\paragraph{
  El arte tiene la particularidad de cada reconfiguración de sus elementos en un instante determinado y/o su transición a la siguiente configuración reasemeja sensaciones conocidas al expectador.
}

\paragraph{
  Si consideremaos que \it{la inspiración} produce de una reconfiguración de elementos previamente conocidos, una creación que surge de la nada es [primer motor]
}


\section{Música}

\paragraph{
  La música es una forma de expresión creada por el ser humano, mediante sonidos intercalados ensambla armonias.
  El silencio sólo existe cómo idea abstracta o en aquellos espacios en los que el sonido no puede viajar, e incluso cuando el sonido no viaja existen objetos que vibran en frecuencias audibles. Un espacio completamente en silencio ha carecer incluso de observador.
}
