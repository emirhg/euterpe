\chapter{Creatividad}
\section{Definicion}
\paragraph{La RAE define creatividad como {\it facultad de creación}}

\paragraph{En el sentido estricto, resulta imposible crear algo de la nada, a lo más podemos trabajar con elementos basicos ya existentes y crear un ensable nuevo de esos elementos, al conjunto resultante, cuandpor la ley de la conservación de la energía, se puede decir que es imposible conocer el proceso creativo mediante el cuál algo surge de la nada. Lo que socialmente conocemos como creatividad humana, es un proceso de razonamiento mediante el cuál un ser humano reconfigura ediante un conjunto de pensamientos elementos prexistentes para darle forma a algo que antes no existía y que tiene un valor social, si el objeto nuevo sólo es apreciable por su creador no es considerado resulado de un proceso creativo.}

\paragraph{Todo trabajo creativo humano es resultado de un trabajo anterior, la música por ejemplo, no existiría si no hubiera un instrumento, el instrumento no produciría sonido si no existiera un material con que moldearlo, el material no seria moldeable si sus atomos no tuvieran determinada configuración, etcétera.}

\paragraph{Vemos que existe una energía fínita en el universo con posibilidades (una n muy grande) infinitas de configuración; entre la luz y la materia sólida existe un espectro finito extremadamente amplio. De esta observación concluimos que la creatividad humana se lleva a cabo dentro de un dominio material finito, con posibilidades incuantificables, pero con una apreciación limitada por el entorno social, no toda combinación tiene un significado para el entorno.}

\paragraph{La música es un arreglo de frecuencias sonoras estandarizadas, cuyo orden y duración producen una pieza musical, definimos el dominio de la música al conjunto de notas musicales reproducibles en un instrumento musical. La diferencia entre la música y el ruido es que el ruido carece de significado, un instrumento musical desafinado puede producir sonidos molestos para el oyente. Lo que para un artísta podría ser una obra maestra, si no transmite al receptor algo con significado no es producto de la creatividad.}

\section{Creatividad Artificial}

El dominio en el que existe la creatividad


Existen muchas deficiones de creatividad, la RAE la define cómo la capacidad de creación, si pensamos ue todo lo que crea el hombre existe ya dentro de un dominio finito


La creatividad la reconocemos por su resultado

\section{Tipos de creatividad}

\paragraph{Existen dos tipos de creatividad, la creatividad histórica y la creatividad psicológica\cite{pensamiento-creativo} }
\paragraph{Algún tipo cuyo nombre y obra no recuerdo  distingue entre creatividad psicológica y creatividad histórica, en dónde la creatividad histórica es aquella idea nueva que nunca se ha hecho antes y la creatividad psicólogica es aquello nuevo en la mente del individuo.}


\begin{comment}
    De existir una máquina capaz de analizar el lenguaje humano, e.i. operar con todo el conocimiento humano disponible y los problemas humano, esa máquina sería capaz de proponer soluciones novedosas a los problemas humanos.

    ¿Es posible crear una máquina que invente un avión a partir de las leyes de la física y el requisito: volar?
\end{comment}


\section {Psicología de la creatividad}
\paragraph {No existe una defición concreta de creatividad, sin embargo todos los intentos por definir la creativid coinciden en que el resultado de la creatividad es la producción de algo que no existia previamente, pero tambien hay un factor social, no todas las creaciones novedosas son creativas, vemos que es necesario el reconocimiento de algún tercero para que la creación sea considerada creativa.}

\paragraph {Consideramos que la creatividad es una expresión de la originalidad de cada individuo, es poco probable que dos individuos creen una copia identica del mismo trabajo cuando se les deja actuar libremente a merced de su imaginación, la imaginación juega un papel importante en la creatividad, es ahi donde se producen ideas nuevas, aunque no todas puedan ser realizables, interviene tambien la inteligencia para poder dicernir lo realizable de las utopias.}

\paragraph {La creatividad de un sistema no se puede medir solamente con base en su salida, hay que tomar en consideración también la forma en la que el artefacto es producido. Entre los consumidores de arte ua obra es mejor valuada cuando el proceso mieante el que se hizo es más creativo. \cite{simon-colton}}

\section{Proceso creativo}

\paragraph{Habilidad, apreciación  e imaginación}
\section{Creatividad en la naturaleza}
\section{Creatividad artificial}


\section{Imitación de estilos}
  3. Codificar el fenotipo de una melodía
  4. Definir el proceso de mutación y cruce
  5. Definir el tamaño de la población
  6. Definir condificación de paro
    ¿Cuánto tiempo tomaría explorar el universo completo de posibilidades?

    Experimentos Set 2:
    1.- Generar un armónimo.
    2.- Generar una secuencia monótona
    3.- Generar una secuencia de armónicos.
    4.- Producir una secuencia monótona imitando un estílo musical
    5.- Producir una secuencia de armónicos imitando un estílo musical.


\section{Presentación}

  Autómata generador de melodías

  \paragraph{Los instrumentos musicales tienen una}
  \paragraph{Si consideramos una cantidad finita de generadores de }
