\section{Objetivo}

Necesito una nueva pieza musical de [Inserte autor aquí], ¿qué debo hacer?" (EdwardBono)
El presente proyecto pretende imitar el estilo musical de un autor, mediente el uso de algoritmos genéticos se pretende generar una nueva pieza musical que sea lo más parecida a un estilo musical determinado.
Se pretende generar una melodía en formato MIDI que comparta características estéticas del estilo de Francisco Tarrega sin que sea confundible con una composición previamente conocida del mismo autor.

Desarrollar un sistema autonomo capaz de componer una pieza musical muy similar al estilo de un compositor especifico.

Es posible crear una pieza musical similar a otras piezas musicales de un autor determinado realizando combinaciones aleatorias en las notas musicales y conservando aquellos conjuntos de notas cuyo arreglo se asemeje en la mayor medida posible sobre la combinación de notas que posiblemente hubiera realizado el autor o estilo que se quiere imitar.

%	\section{Definición del problema}
%
%	\section{Alcance}

%		\item ¿Que elementos tienen en común el estilo y el genero?
%		\item ¿Cómo se define matematicamenet el dominio de la música?
%		\item ¿Puede diferenciarse el estilo musical si todas las notas se tocan en la misma octava?
%
%		\item ¿Qué es un beat?
%			Un beat mide en segundos la duración de la figura musical (tambien llamada nota) de menor valor en una pieza musical.
%
%		\item ¿Qué es la creatividad humana?
%				Es la facultad que tiene el humano de crear
%				La creatividad humana se conoce cuando se entra en contacto con ella, debe ser novedoso y entendible.
%
