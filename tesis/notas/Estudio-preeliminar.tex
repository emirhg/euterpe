\chapter{Creatividad}
\section{Definicion}
\paragraph{La RAE define creatividad como {\it facultad de creación}}

\paragraph{En el sentido estricto, resulta imposible crear algo de la nada, a lo más podemos trabajar con elementos basicos ya existentes y crear un ensable nuevo de esos elementos, al conjunto resultante, cuandpor la ley de la conservación de la energía, se puede decir que es imposible conocer el proceso creativo mediante el cuál algo surge de la nada. Lo que socialmente conocemos como creatividad humana, es un proceso de razonamiento mediante el cuál un ser humano reconfigura ediante un conjunto de pensamientos elementos prexistentes para darle forma a algo que antes no existía y que tiene un valor social, si el objeto nuevo sólo es apreciable por su creador no es considerado resulado de un proceso creativo.}

\paragraph{Todo trabajo creativo humano es resultado de un trabajo anterior, la música por ejemplo, no existiría si no hubiera un instrumento, el instrumento no produciría sonido si no existiera un material con que moldearlo, el material no seria moldeable si sus atomos no tuvieran determinada configuración, etcétera.}

\paragraph{Vemos que existe una energía fínita en el universo con posibilidades (una n muy grande) infinitas de configuración; entre la luz y la materia sólida existe un espectro finito extremadamente amplio. De esta observación concluimos que la creatividad humana se lleva a cabo dentro de un dominio material finito, con posibilidades incuantificables, pero con una apreciación limitada por el entorno social, no toda combinación tiene un significado para el entorno.}

\paragraph{La música es un arreglo de frecuencias sonoras estandarizadas, cuyo orden y duración producen una pieza musical, definimos el dominio de la música al conjunto de notas musicales reproducibles en un instrumento musical. La diferencia entre la música y el ruido es que el ruido carece de significado, un instrumento musical desafinado puede producir sonidos molestos para el oyente. Lo que para un artísta podría ser una obra maestra, si no transmite al receptor algo con significado no es producto de la creatividad.}

\section{Tipos de creatividad}

\paragraph{Existen dos tipos de creatividad, la creatividad histórica y la creatividad psicológica\cite{pensamiento-creativo} }


\begin{comment}
    De existir una máquina capaz de analizar el lenguaje humano, e.i. operar con todo el conocimiento humano disponible y los problemas humano, esa máquina sería capaz de proponer soluciones novedosas a los problemas humanos.
    
    ¿Es posible crear una máquina que invente un avión a partir de las leyes de la física y el requisito: volar?
\end{comment}


\chapter{Introducción}

\begin{comment}
Tesis. 1. f. Conclusión, proposición que se mantiene con razonamientos.


Creatividad.
1. f. Facultad de crear.
2. f. Capacidad de creación.

Artificial
(Del lat. artificiālis).
1. adj. Hecho por mano o arte del hombre.
2. adj. No natural, falso.
3. adj. Producido por el ingenio humano.
4. adj. ant. artificioso (disimulado, cauteloso).
\end{comment}

\section{Creatividad Artificial}

El dominio en el que existe la creatividad


Existen muchas deficiones de creatividad, la RAE la define cómo la capacidad de creación, si pensamos ue todo lo que crea el hombre existe ya dentro de un dominio finito


La creatividad la reconocemos por su resultado



\section{Objetivo}

Desarrollar un sistema autonomo capaz de componer una pieza musical muy similar al estilo de un compositor especifico. 

Es posible crear una pieza musical similar a otras piezas musicales de un autor determinado realizando combinaciones aleatorias en las notas musicales y conservando aquellos conjuntos de notas cuyo arreglo se asemeje en la mayor medida posible sobre la combinación de notas que posiblemente hubiera realizado el autor o estilo que se quiere imitar.



\section{Antecedentes}
- Creatividad artificial
- Imitación computarizada

\section{Definición de creatividad}
Composición de música con intervención mínima humana

La nuevade la computación tiene una fuerte tendencia a replicar carácteristicas humanas.

En su afan del hombre por entender los procesos naturales y usarlos a su beneficio a incursionado en áreas muy novedosas en el área de la computación.

La inteligencia artificial es uno de lso intentos del hombre por construir una réplica de si mismo, utilizando los conocimientos que tiene sobre el funcionamiento del proceso cognitivo.

La creativididad sigue siendo un concepto vago, deficil de delimitar, 

Para que algo sea considerado creativo tiene que ser lógico a posteriorí, de lo contrario es sólo una idea alocada, sin sentido. El hombre cómo ser racional es el que juzga si algo es creativo o no, para que sea considerado creativo tiene que ser novedoso y lógico.

\subsection{Hipotesis}


- La creatividad es un proceso
- La creatividad se ejercita
- No todas las obras artísticas son creativas
- Algo creativo es nuevo y tiene valor


\section{Comparación entre la creatividad humana y la de las aves}

    Música. El canto de un pájaro y las melodías compuestas por seres humanos
    Arquitectura. Los edificiós para hábitad humano y los nidos de las aves
    
Otras áreas creativas propias del ser humano:
    Pintura
    Escultura
    Vestido (diseñadores de moda)
    

\section{Definición del problema}
Necesito una nueva pieza musical de [Inserte autor aquí], ¿qué debo hacer?" (EdwardBono)


\section{Dudas}

¿Existe arte que no sea creativo?

¿Cuál es la diferencia entre arte y creatividad?