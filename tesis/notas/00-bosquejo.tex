Cada ser humano es libre de tomar desiciones

\chapter{Introducción}


"El ácido desoxirribonucleico, abreviado como ADN, es un ácido nucleico que contiene las instrucciones genéticas usadas en el desarrollo y funcionamiento de todos los organismos vivos conocidos y algunos virus, y es responsable de su transmisión hereditaria." \cite{wikipedia-adn-2016-01-11}

\paragraph{Los instromentos musicales tienen una}
\paragraph{Si consideramos una cantidad finita de generadores de }

\section{justificación}
\section{antecedentes}
\section{materiales y métodos}
\section{resultados y discusión}


\section{Preguntas}

¿Es una buena pregunta?

¿Ya ha sido contestada? 
¿Es unapregunta en la que vale la pena trabajar?
- ¿Cómo grabar los sueños?


¿Existe alguna relación entre la inteligencia  y la creatividad?

¿Cómo se diferencía el ruido de la música?

¿Que es una melodía?
¿Cómo se define un estílo musical?
¿Qué diferencía a un compositor de otro?
¿Qué es un estilo musical?
¿Puede diferenciarse el estilo musical si todas las notas se tocan en la misma octava?
¿Que es el género musical?

¿Que elementos tienen en común el estilo y el genero?
¿Cómo se define matematicamenet el dominio de la música?

¿Qué es un beat?
	Un beat mide en segundos la duración de la figura musical (tambien llamada nota) de menor valor en una pieza musical.

¿Cuántos beats existen en una grabación de duración T?
	Si un beat tiene una duración t, 


¿Se puede replicar para mejorar la naturaleza?
No

\begin{equation} \label{eq:n}
	n = \frac{T}{t}
\end{equation}

El número de beats n se obtiene mediante la fórmula \eqref{eq:n}


\section{El formato MIDI}
	El formato MIDI permite representar en forma digital las notas tocadas por un instrumento musical en forma de eventos. El un MIDI puede representar hasta 16 canales, cada uno asociado a un instrumento. Cuando un instrumento sólo puede producir una sola nota, se dice que es un instrumentono monótono


	

%\paragraph{
%>> [max(tarreaga);min(tarreaga);max(tarreaga)-min(tarreaga)]

%ans =

  379.6250    3.0000    2.0000   88.0000  100.0000  197.5583    1.9167
         0    0.1250    1.0000   40.0000  100.0000         0    0.0625
  379.6250    2.8750    1.0000   48.0000         0  197.5583    1.8542
%}
  
\section{Representación del modelo de entrenamiento}
    \paragraph{
        hi
    }