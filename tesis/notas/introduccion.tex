\section{Introducción}

%  Glosario
%     Tesis.
%     1. f. Conclusión, proposición que se mantiene con razonamientos.
%
%     Creatividad.
%     1. f. Facultad de crear.
%     2. f. Capacidad de creación.
%
%     Artificial
%     (Del lat. artificiālis).
%     1. adj. Hecho por mano o arte del hombre.
%     2. adj. No natural, falso.
%     3. adj. Producido por el ingenio humano.
%     4. adj. ant. artificioso (‖ disimulado, cauteloso).
%
%     Arte
%     Del lat. ars, artis, y este calco del gr. τέχνη téchnē.
%     1. m. o f. Capacidad, habilidad para hacer algo.
%     2. m. o f. Manifestación de la actividad humana mediante la cual se interpreta lo real o se plasma lo imaginado con recursos plásticos, lingüísticos o sonoros.
%     3. m. o f. Conjunto de preceptos y reglas necesarios para hacer algo.
%     4. m. o f. Maña, astucia.
%     5. m. o f. Disposición personal de alguien. Buen, mal arte.
%     6. m. o f. Instrumento que sirve para pescar. U. m. en pl.
%     7. m. o f. rur. Man. noria (‖ máquina para subir agua).
%     8. m. o f. desus. Libro que contiene los preceptos de la gramática latina.
%     9. m. o f. pl. Lógica, física y metafísica. Curso de artes.

%     Músico, ca
%     Del lat. musĭcus, y este del gr. μουσικός mousikós; la forma f., del lat. musĭca, y este del gr. μουσική mousikḗ.
%     1. adj. Perteneciente o relativo a la música. Instrumento músico. Composición música.
%     2. m. y f. Persona que conoce el arte de la música o lo ejerce, especialmente como instrumentista o compositor.
%     3. m. Cuba. faurestina.
%     4. f. Melodía, ritmo y armonía, combinados.
%     5. f. Sucesión de sonidos modulados para recrear el oído.
%     6. f. Concierto de instrumentos o voces, o de ambas cosas a la vez.
%     7. f. Arte de combinar los sonidos de la voz humana o de los instrumentos, o de unos y otros a la vez, de suerte que produzcan deleite, conmoviendo la sensibilidad, ya sea alegre, ya tristemente.
%     8. f. Compañía de músicos que cantan o tocan juntos. La música de la Capilla Real.
%     9. f. Composición musical. La música de esta ópera es de tal autor.
%     10. f. Colección de papeles en que están escritas las composiciones musicales. En este escritorio se guarda la música de la capilla.
%     11. f. Sonido grato al oído. La música del viento entre las ramas. La música del agua del arroyo.
%     12. f. irón. Ruido desagradable.
%     13. f. coloq. música celestial.

%     Estilo
%     Del lat. stilus 'punzón para escribir', 'modo de escribir'.
%     1. m. Modo, manera, forma de comportamiento. Tiene mal estilo.
%     2. m. Uso, práctica, costumbre, moda.
%     3. m. Manera de escribir o de hablar peculiar de un escritor o de un orador. El estilo de Cervantes.
%     4. m. Carácter propio que da a sus obras un artista plástico o un músico. El estilo de Miguel Ángel. El estilo de Rossini.
%     5. m. Conjunto de características que identifican la tendencia artística de una época, o de un género o de un autor. Estilo neoclásico.
%     6. m. Gusto, elegancia o distinción de una persona o cosa. Pepa viste con estilo.
%     7. m. Punzón con el cual escribían los antiguos en tablas enceradas.
%     8. m. gnomon (‖ indicador de las horas en el reloj solar).
%     9. m. Bot. Columna pequeña, hueca o esponjosa, existente en la mayoría de las flores, que arranca del ovario y sostiene el estigma.
%     10. m. Dep. Cada una de las distintas formas de realizar un deporte. Prueba en estilo mariposa.
%     11. m. Mar. Púa sobre la cual está montada la aguja magnética.
%     12. m. Arg. y Ur. Composición musical de origen popular, para guitarra y canto, de carácter evocativo y espíritu melancólico.
%     13. m. Ur. Baile popular que se ejecuta con el estilo.

\paragraph{Cuando se habla del estilo musical se hace referencia al conjunto de características acústicas que permiten asociar una pieza musical con un género, autor o época\cite{rae-estilo} }

%
%		\item ¿Existe arte que no sea creativo?
%
%		\item ¿Cuál es la diferencia entre arte y creatividad?
%
%		\item ¿Existe alguna relación entre la inteligencia  y la creatividad?
%
%		\item ¿Cómo se diferencía el ruido de la música?
%
%		\item ¿Que es una melodía?
%		\item ¿Cómo se define un estílo musical?
%		\item ¿Qué diferencía a un compositor de otro?
%		\item ¿Qué es un estilo musical?
%		\item ¿Que es el género musical?
%

%  \paragraph{La creatividad humana es una mezcla de memorias, interlazadas de forma coherente, cuya expresión es novedosa y tiene significado dentro de un contexto colectivo especifico, cuando la interpretación de esta expresión de memorias es incomprencible por agentes externos dentro del mismo contexto, se habla de irracionalidad o locura y no de creatividad.}
%
%  \paragraph{La creatividad y el arte son resultado de cómo el cerebro humano opera con los datos percibidos por los organos sensoriales. Podemos reconocer momentos creativos en sueños, mientras estamos absortos en algún pensamiento concreto o cuanto "de la nada" nos llega una idea aperentemente sin relación con la actvidad actual. En este proyecto se pretende exponer una teoria computacional de cómo opera el pensmiento creativo para crear ideas nuevas a partir de ideas conocidas.}
%
%
%  \section{Definición de creatividad}
%  Composición de música con intervención mínima humana
%
%  La nuevade la computación tiene una fuerte tendencia a replicar carácteristicas humanas.
%
%  En su afan del hombre por entender los procesos naturales y usarlos a su beneficio a incursionado en áreas muy novedosas en el área de la computación.
%
%  La inteligencia artificial es uno de lso intentos del hombre por construir una réplica de si mismo, utilizando los conocimientos que tiene sobre el funcionamiento del proceso cognitivo.
%
%  La creativididad sigue siendo un concepto vago, deficil de delimitar,
%
%  Para que algo sea considerado creativo tiene que ser lógico a posteriorí, de lo contrario es sólo una idea alocada, sin sentido. El hombre cómo ser racional es el que juzga si algo es creativo o no, para que sea considerado creativo tiene que ser novedoso y lógico.
%
%
%
%  \section{Comparación entre la creatividad humana y la de las aves}
%
%      Música. El canto de un pájaro y las melodías compuestas por seres humanos
%      Arquitectura. Los edificiós para hábitad humano y los nidos de las aves
%
%  Otras áreas creativas propias del ser humano:
%      Pintura
%      Escultura
%      Vestido (diseñadores de moda)


% HIPOTESIS


% - La creatividad es un proceso
% - La creatividad se ejercita
% - No todas las obras artísticas son creativas
% - Algo creativo es nuevo y tiene valor
%
%
%
%     Hipótesis: Creatividad a partir de la destrucción.
%         Cómo un escultor dá forma al marmol mediante la destrucción, se puede crear armonía a partir del ruido.
%
%     Hipótesis: Creación acúmulativa.
%         Similar a la creación literaria; conjugar e hilar conceptos conocidos.
%
%     Hipótesis: Creatividad evolutiva.
%         Creación y destrucción coexisten en la evolución.
%
