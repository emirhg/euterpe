\section{El formato MIDI}
	El formato MIDI permite representar en forma digital las notas tocadas por un instrumento musical en forma de eventos. El un MIDI puede representar hasta 16 canales, cada uno asociado a un instrumento. Cuando un instrumento sólo puede producir una sola nota, se dice que es un instrumentono monótono

\begin{verbatim}
>> [max(tarreaga);min(tarreaga);max(tarreaga)-min(tarreaga)]

ans =

	379.6250    3.0000    2.0000   88.0000  100.0000  197.5583    1.9167
				 0    0.1250    1.0000   40.0000  100.0000         0    0.0625
	379.6250    2.8750    1.0000   48.0000         0  197.5583    1.8542
\end{verbatim}

\section {Conteo del dominio del problema}
  Definir el dominio del problema (espacio musical)
    ¿Cuantas posibilidades existen para componer una melodía?

    Las posibilidades de la música extrictamente hablando son infinitas, sin embargo podemos acotar el dominio acotando su extensión temporal y la cantidad de notas a elegir

    ¿Cuántos beats existen en una grabación de duración T?
      Si un beat tiene una duración t,

		El número de beats n se obtiene mediante la fórmula \eqref{eq:n}

			\begin{equation} \label{eq:n}
				n = \frac{T}{t}
			\end{equation}
