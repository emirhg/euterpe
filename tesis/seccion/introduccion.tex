\chapter{Introducción}
\begin{comment}

Aparte de que faltarían muchos capítulos, en las partes que escribiste hay mucho “qué” (es decir, explicaciones de lo que se hizo), pero falta mucho “por qué” (explicaciones/justificaciones de por qué se hizo, y de por qué se hizo de la manera en la que se hizo y no de cualquier otra manera).

¿Puede una computadora crear algo útil para el ser humano con un universo de elementos previamente provistos por el humano?.
Las tragedias griegas originalmente iban representadas con canto y danza, y se consideraba la poesía, la musica y la danza un solo arte.

¿Cómo surge el sentido de la belleza en el ser humano?

\end{comment}

\begin{comment}
\section{sin nombre}

Una computadora es probablemente la herramienta más útil con la que cuenta el ser humano hoy día, nos ha permitido ir a la luna, mejorar nuestros sistemas de comunicaciones, crear prótesis biónicas

En la actualidad existe una especie de simbiosis humano-máquina cómo probablemente no ha existido en la historia entre el humano y otro ser vivo, al menos de forma conciente. Si bien el ser humano se ha beneficiado mucho de las máquinas no podemos decir que una máquina se haya beneficiado de un humano, responder si una actualización de software o hardware significa un beneficio para la máquina está fuera del proposito de este trabajo, sin embargo asumiremos como premisa el hecho de que una computadora carece de conciencia y que el beneficio o perjucio sólo existen en seres concientes.

\end{comment}

\section{¿Qué es la creatividad?}
\paragraph{Se entiende como creatividad la facultad de crear algo novedoso, ideas, conceptos u objetos tangibles que tengan alguna utilidad. Esta caracteristica de utilidad es lo que diferencia una idea creativa de una tonta.}

\section{¿Qué es la creatividad artificial?}
\paragraph{La creatividad artificial es un esfuerzo multidisciplinario en el que intervienen distintas áreas tales como Inteligencia Artificial, Filosofía, Ciencias Cognitivas y Artes. El objetivo es crear un software que replique la creatividad tal cómo ocurre en el nivel humano.}

\section{Objetivo}
\paragraph{En esta tesis se analizan las características de una obra musical y se plantea la posibilidad de que una computadora cree una obra musical que comparta características sonoras con otras obras previas de un mismo artista. Partiendo del hecho de que una obra puede ser atribuida a un artista por su estilo se describe un modelo probabilístico del estilo y mediante el uso de algoritmos genéticos se desea crear una obra que comparta las características de ese estilo.}

\begin{comment}
  TODO: Preguntas que inmediatamente saltan a la mente del lector (y que habría que responder aquí mismo): ¿por qué algoritmos genéticos?  ¿No hay otras formas posibles de hacerlo?  En caso de que las haya, ¿los algoritmos genéticos tienen alguna ventaja sobre las demás posibles formas de lograrlo?
\end{comment}

\paragraph{La presente tesis aborda el proceso creativo como combinación de elementos existentes en el entorno con la finalidad de crear algo que no existía antes en el entorno. Hasta la fecha no existe un modelo que replique a la perfección el proceso creativo de un humano.}

\begin{comment}
  REVIEW: Pero sí ha habido otros esfuerzos...habría que mencionar un puñado de ejemplos (tanto musicales como de otros ámbitos), citando publicaciones y mencionando qué hacen (y qué les falla o les falta).
\end{comment}

\paragraph{Se entiende como proceso creativo la mezcla de memorias, combinadas de forma coherente, cuya expresión es novedosa y tiene significado dentro de un contexto colectivo específico. Cuando la interpretación de esta expresión de memorias es incomprensible por agentes externos dentro del mismo contexto, se dice que la idea es incoherente.}

\begin{comment}
  REVIEW: ¿No que es una mezcla (no una expresión) de memorias?
\end{comment}

\paragraph{La creatividad y el arte son resultado de cómo el cerebro humano opera con los datos percibidos por los órganos sensoriales. Podemos reconocer momentos creativos en sueños, mientras estamos absortos en algún pensamiento concreto o cuando ``de la nada'' nos llega una idea aperentemente sin relación con la actvidad actual.}

\paragraph{Algún tipo cuyo nombre y obra no recuerdo\cite{required}
% TODO: Pero hay que investigarlo e incluir el dato...  Además es "una tipa" (Margaret Boden).
distingue entre creatividad personal
y creatividad histórica, en donde la creatividad histórica es aquella idea nueva que nunca se ha propuesto antes y la creatividad personal es aquello nuevo en la mente del individuo.
% REVIEW: ¿Cuál?
}


\paragraph{El presente proyecto pretende imitar el estilo musical de un autor. Mediente el uso de algoritmos genéticos se pretende generar una nueva pieza musical que sea lo más parecida a un estilo musical determinado.}

