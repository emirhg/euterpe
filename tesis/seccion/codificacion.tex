\chapter{Clasificación de estilos musicales}

\section{Conceptos básicos}
\paragraph{Empezaremos por definir algunos conceptos básicos relacionados con la música.}

%\subsection{Percepción}

\subsection{Sonido}
    \paragraph{El ser humano es capaz de escuchar por medio del oido como consecuencia de las vibraciones que se reproducen en el tímpano. Estás vibraciones son producto de la fuerza ejercida por el aire en el organo. Cuando se ejerce una presión en un medio elastico tal que este oscile, se dice que las ondas producidas por el movimiento del material son sonido. Por tratarse de un organo elástico, el tímpano se puede estirar, produciendo en el una oscilación entre dos puntos. El tiempo durante el cuál el timpano es desplazado de su estado de reposo hasta un punto $a$, libera su energia potencial, alcanza un mínimo $b$ y regresa a el punto dónde solia estar en reposo se conoce cómo periodo, y oscilación o ciclo al movimiento descrito el cuál tiene una forma senoidal. Utilizamos los Hertz para medir las oscilacianes que se producen en un segundo, i.e., cuando la oscilación completa se produce en un lapso de 1 segundo, se dice que la señal u onda sonora que llega al timpano tiene una frecuencia $f = 1 Hz$. La función $$f(f_0) = sin(2*\pi*f_0*t)$$ describe una onda que oscila a una frecuencia $f_0$. Las limitaciones fisiológicas del organo auditivo sólo nos permiten percibir las oscilaciones que ocurren entre 20 Hz y 20KHz, este rango de frecuencias se conoce como espectro audible y podemos definirlo cómo el conjunto $$ E_a = [20, 20000] Hz$$}
    
    
    \paragraph{En el mundo real los sonidos que escuchamos están compuestos por multiples frecuencias, sino que están compuestos de multiples frecuencias. En adelante cuando hablemos de sonidos nos referiremos a estos sonidos compuestos audibles por el ser humano.}
    
    \paragraph{En teoria musical se dice que el sonido tiene 4 propiedades: la altura o tono; la intensidad, el timbre o color; y la duración. El tono es una característica psicoacústica del sonido, se refiere a la distinción entre un sonido grave y uno agudo, mientras que la frecuencia es una medida objetiva del sonido, el tono es subjetivo, sin embargo existen estandares para nombrar los tonos según su frecuencia más baja, llamada frecuencia fundamental; el color se refiere a las frecuencias que se agregan al sonido cómo resultado de la resonancia de la frecuencia fundamental dentro del amplificador sonoro, es lo que nos permite diferenciar al emisor de un sonido con la misma frecuencia fundamental, un \textbf{La} tiene un timbre diferente si es interpretado en una flauta o en piano; la intensidad indica la fuerza con la que un sonido es producido; y la duración el intervalo de tiempo durante el cuál existe.}


    \paragraph{...}
    
    \paragraph{Un acorde es un sonido compuesto por muchos tonos simultaneos. Vemos que podemos describir la música cómo una secuencia de silencios, tonos y acordes.}

    \paragraph{...}
    
    \paragraph{Se propone modelar la música cómo una Cadena de Markov en la que cada estado representa un sonido.}

\subsection{MIDI}
Un archivo MIDI permite grabar y reproducir los eventos acústicos de un instrumento musical. El MIDI almacena la información de un instrumento en mensajes de eventos, cada evento contiene información de la frecuencia $f$ y posición temporal $t$ de la nota interpretada con un instrumento musical (que cuente con interfaz para MIDI).